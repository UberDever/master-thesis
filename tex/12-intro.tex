\Introduction

Особенностью современной индустрии разработки программного обеспечения является применение нескольких языков программирования в
одном программном проекте. Данная особенность порождает проблему проверки согласованности фрагментов кода, реализованных на разных языках и
применяемых в составе одной программной системы. Как правило, такая несогласованность отслеживается только на этапе отладки или тестирования.

Одним из вариантов решения данной проблемы может стать проведение статического семантического анализа кода
с выявлением несогласованности в таком коде автоматически. Для решения такого рода задач существуют различные
инструментальные средства и часто такие средства встраиваются в среду разработки для
поддержки разработчика в реальном времени.

Однако, на данный момент методы проведения статического анализа мультиязыкового
кода развиты слабо. Большая часть методов ориентирована на специфические языки или проекты
и плохо подходит для интеграции в инструментальные средства в общем случае.
Вследствие этого мультиязыковые средства поддержки разработчика слабо развиты
и обычно ограничены кругом узкоспециализированных проприетарных решений.

Метод мультиязыкового анализа, позволяющий гибко анализировать
фрагменты кода на разных языках и независимый относительно анализируемых
технологий является одним из решений данной проблемы.

\textbf{Актуальность темы исследования} обусловлена малым развитием
инструментальных средств поддержки разработчика, ориентированных на мультиязыковой анализ.
Такое положение вещей снижает продуктивность разработчиков и увеличивает время, отводимое
на процессы кодирования и отладки.

\textbf{Объектом исследования} являются исходные тексты и семантические модели
программ, написанных с использованием нескольких языков
программирования.

\textbf{Предметом исследования} являются методы и алгоритмы анализа исходных
текстов программ, ориентированных на мультиязыковой статический анализ.

\textbf{Целью работы} является упрощение процессов разработки ПО, путем создания метода выявления межъязыковых связей,
позволяющего обеспечивать корректность и расширяемость соответствующих средств.

Для достижения данной цели необходимо решить следующие \textbf{задачи}:
\begin{enumerate}[1)]
    \item изучить предметную область, а именно индустрию разработки ПО,
    \item провести исследование текущих решений в области межъязыкового анализа,
    \item разработать требования к методу и спроектировать его архитектуру,
    \item описать прикладные сценарии использования метода и соответствующего анализатора,
    \item создать первичную реализацию анализатора, поддерживающего определенный набор языков,
    \item провести оценку показателей анализатора на избранных тестовых проектах.
\end{enumerate}

\textbf{Научная новизна} работы заключается в создании метода анализа, позволяющего
разрабатывать различные инструментальные средства, направленные
на поддержку разработчика в процессе кодирования и отладки. При этом, такой метод позволит
обеспечить корректность семантического анализа на межъязыковом уровне.

\textbf{Практическая значимость} работы выражается в создании прототипа мультиязыкового
анализатора, обеспечивающего поддержку реализации сценариев LSP, что позволяет
создавать <<мультиязыковые>> серверы LSP.

\Define{Семантика}{смысловое содержание синтаксических конструкций программ}