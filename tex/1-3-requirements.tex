\chapter{Постановка задачи, формирование требований к методу}

\section{Цели и задачи}

Цель данной работы: Упрощение процессов разработки ПО, путем создания метода выявления межъязыковых связей,
позволяющего обеспечивать корректность и расширяемость соответствующих средств.

Для достижения данной цели необходимо решить следующие задачи:
\begin{enumerate}[1)]
    \item изучить предметную область, а именно индустрию разработки ПО,
    \item провести исследование существующих решений в области межъязыкового анализа,
    \item разработать требования к реализации и её программную архитектуру,
    \item описать прикладные сценарии использования метода и соответствующего анализатора,
    \item создать прототип анализатора, поддерживающего определенный набор языков,
    \item выполнить оценку показателей анализатора на избранных тестовых проектах.
\end{enumerate}


\section{Требования к методу анализа}

В требования к методу анализа входит:
\begin{enumerate}[1)]
    \item возможность обнаружения межъязыковых зависимостей (в рамках фрагмента кода),
    \item независимость в отношении анализируемых языков и технологий,
    \item возможность гибкой настройки объемов анализа,
    \item использование семантического представления с формально доказанными свойствами,
    \item ориентированность на основные сценарии использования средств
    поддержки разработчика. 
\end{enumerate}

\clearpage