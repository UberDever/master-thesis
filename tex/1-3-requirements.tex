\chapter{Постановка задачи, формирование требований к методу}

\section{Цели и задачи}

Цель данной работы: <<Упрощение процессов разработки ПО, путем создания метода выявления межъязыковых связей,
позволяющего обеспечивать корректность и расширяемость соответствующих средств.>>

Для достижения данной цели необходимо решить следующие задачи:
\begin{enumerate}[1)]
    \item изучить предметную область, а именно индустрию разработки ПО,
    \item провести исследование текущих решений в области межъязыкового анализа,
    \item разработать требования к методу и спроектировать его архитектуру,
    \item описать прикладные сценарии использования метода и соответствующего анализатора,
    \item создать первичную реализацию анализатора, поддерживающего определенный набор языков,
    \item провести оценку показателей анализатора на избранных тестовых проектах.
\end{enumerate}


\section{Требования к методу анализа}

В основные требования к методу анализа входит:
\begin{enumerate}[1)]
    \item возможность обнаружения межъязыковых зависимостей (в рамках фрагмента кода),
    \item использование выходного представления данных анализа, отражающего ключевые концепции
    языков программирования,
    \item обеспечение корректности анализа в отношении семантики исходных языков,
    \item возможность предоставления наиболее полных результатов анализа в рамках
    избранного выходного представления,
    \item ориентированность на основные сценарии использования средств
    поддержки разработчика.
\end{enumerate}

В обеспечивающие требования к методу анализа входит:
\begin{enumerate}[1)]
    \item использование общепринятого формата данных для передачи результатов анализа,
    \item гибкость в отношении количества анализаторов и количества анализируемой информации,
    \item минимизация ложноположительных результатов анализа,
    \item возможность обеспечения быстрого отклика использующих метод средств.
\end{enumerate}


\clearpage